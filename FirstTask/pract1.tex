\section*{\Large Цель практическои работы}
\addcontentsline{toc}{section}{Цель практическои работы}

\textbf{Цель работы} --- изучить структуру и функционал рассматриваемой
информационной системы, освоить правила посторения диаграммы вариантов
использования.

\textbf{Задачи:}\par
\begin{itemize}
	\item изучить предметную область по заданным вариантам;
	\item определить на концептуальном уровне состав элементов системы;
	\item описать функции рассматриваемой системы с помощью диаграммы 
		вариантов использования.
\end{itemize}
\newpage

\section{Анализ существующих сайтов}
\subsection{Аптека 36.6}

\textbf{Достоинства:}\par
\begin{itemize}
	\item есть поисковая строка;
	\item есть регистрация;
	\item есть возможность помечать товары как избранные;
	\item товар разбит на категории, такие как:
	\begin{itemize}
		\item выгодно;
		\item лекарства;
		\item витамины и БАДы;
		\item косметика;
		\item гигиена;
		\item уход за больными;
	\end{itemize}
	\item есть программа лояльности;
	\item есть меню с инструкцией, как заказать товар;
\end{itemize}

\textbf{Недостатки:} нет выбора товара по производителю,
нет возможности сортировки товара по производителю,
нет возможности выбора товара по сроку годности,
нет возможности выбора товара по наличию.

\subsection{Горздрав}
\textbf{Достоинства:}\par
\begin{itemize}
	\item возможность поиска товара по алфавиту;
	\item есть регистрация;
	\item есть возможность помечать товары как избранные;
	\item товар разбит на категории, такие как:
	\item вкладка со скидками и акциями
\end{itemize}

\textbf{Недостатки:} нет возможности выбора товара по производителю,
нет оформить подписку на ежемесечную доставку товара. Также поля для получения информации о стадии выполнения заказов.

\section{Основные необходимы функции}
На основе проведнного исследования предметной области были выделены основные
функции, которые должна выполнять система. Эти функции будут использоваться
при построении диаграммы вариантов использования.

\textbf{Функции:}
\begin{itemize}
	\item Поиск товара (фильтры): по названию; по алфавиту; по производителю;
		по сроку годности; по наличию; по категории; по цене; по рейтингу;
	\item Возможность оформить ежемесячную подписку на доставку товара;
	\item Регистрация личного кабинета для сохранния истории заказов;
	\item Поле для получения информации о стадии выполнения заказов;
	\item Возможность пометить товар как избранный;
	\item Поле для оставления отзывов о товаре;
	\item База данных, хранящая информацию о клиентах (ФИО, телефон и др.);
	\item Программа лояльности;
	\item Страницы с различной информацией по теме сайта;
	\item Страницы с инструкцие по заказу товаров;
	\item Возможность выбирать способы оплаты;
	\item Доставка заказанного товара;
\end{itemize}

На основе выделеных функций опишем решение
в виде таблицы \ref{table:desc_of_elements}.
\begin{table}[h!tp]
	\caption{\leftline{Описание элементов}}
	\label{table:desc_of_elements}
	\begin{tabular}{|p{0.25\textwidth}|p{0.70\textwidth}|}
		\hline Наименование & Краткое описание\\ \hline
		Выбор товара & Поиск товара(фильтры).\par
			Возможность пометить товар как избранный.\\ \hline
		Регистрации & Регистрация личного
			кабинета для сохранния истории заказов.\\ \hline
		Обратный отклик & Поле для оставления отзывов о товаре.\\ \hline
		Подписка & Программа лояльности.\par
			Возможность оформить ежемесячную
			подписку на доставку товара.\\ \hline
		Предоставление информации & Страницы с различной информацией
			по теме сайта.\par Страницы с инструкцие по заказу товаров.\par
			Поле для получения информации о стадии выполнения заказов.\\ \hline
		Сохранение данных о пользователях & База данных, хранящая информацию
			о клиентах (ФИО, телефон и др.).\\ \hline
		Оплата & Возможность выбирать способы оплаты\\ \hline
		Доставка & Доставка заказанного товара.\\ \hline
	\end{tabular}
\end{table}

\section{Ожидаемые результаты реализации}
По итогу, ожидаемые результаты реализации моделируемой
системы включают: уменьшение времени поиска товара,
увеличение количества привлеченных клиентов, увеличение прибыли, 
уменьшение скорости предоставления нужной информации.


\section{Диаграмму вариантов использования}
На основе выделеных основных вариантов использования (функций) системы
спроектируем диаграму вариантов использования
(Рисунок~\ref{fig:use_case_diagram}). А затем
для каждого из них построить диаграммы декомпозиции (детализации). 
Также нужно учесть какие активные субъекты должны взаимодействовать с
будущей системой.

\begin{figure}[h!tp]
	\centering
	\includegraphics[width=0.7\textwidth]{use_case_diagram}
	\caption{Диаграмма вариантов использования}
	\label{fig:use_case_diagram}
\end{figure}

\newpage
Далее, добавим к деаграмме актеров
(Рисунок~\ref{fig:use_case_diagram_with_acter}).

\begin{figure}[h!tp]
	\centering
	\includegraphics[width=0.7\textwidth]{use_case_diagram_with_acter}
	\caption{Диаграмма вариантов использования}
	\label{fig:use_case_diagram_with_acter}
\end{figure}

Теперь можно рассатвить связи между элемнтами,
как показанно на рис.~\ref{fig:use_case_complite}.

\begin{figure}[h!tp]
	\centering
	\includegraphics[width=0.7\textwidth]{use_case_complite}
	\caption{Диаграмма вариантов использования}
	\label{fig:use_case_complite}
\end{figure}

\newpage

\section*{Ответы на вопросы}
\addcontentsline{toc}{section}{Ответы на вопросы}

\textbf{Для чего используется язык UML?}\par
UML является языком широкого профиля, это --- открытый стандарт,
использующий графические обозначения для создания абстрактной модели системы,
называемой UML-моделью. UML был создан для
\textbf{\textit{определения, визуализации, проектирования и документирования}},
в основном, программных систем. UML не является языком программирования,
но на основании UML-моделей возможна генерация кода.

\textbf{Какие диаграммы входят в состав языка UML?}\par
Диаграмма активности, диаграмма классов, диаграмма последовательности,
диаграмма состояний, диаграмма компонентов и диаграмма действий.

\textbf{В чем смысл варианта использования?}\par
Описывает, какой функционал разрабатываемой программной системы доступен каждой группе пользователей.

\textbf{Каково назначение диаграмм вариантов использования?}\par
Главное назначение диаграммы вариантов использования заключается
в \textbf{\textit{формализации функциональных требований к системе}} с помощью
понятий соответствующего пакета и возможности согласования полученной
модели с заказчиком на ранней стадии проектирования.

\textbf{Что такое «действующее лицо»?}\par
Под действующим лицом или актер понимается любой объект, субъект или система,
взаимодействующая с моделируемой системой извне.

\textbf{Какие элементы содержит диаграмма вариантов использования?}\par
Диаграмма вариантов использования \textit{состоит} из актеров, для которых
система производит действие, и собственно действия Use Case, которое
описывает то, что актер хочет получить от системы. Дополнительно в
диаграммы могут быть добавлены комментарии

\textbf{Что такое актер?}\par
\textbf{\textit{Актером}} (действующим лицом, актантом, актором)
называется любой объект, субъект или система,
взаимодействующая с моделируемой системой извне.

\textbf{Какие отношения возможны между актерами?}\par
Обобщения, включения.

\newpage

\section*{Вывод}
\addcontentsline{toc}{section}{Вывод}
В результате проделанной работы был проведен анализ предметной области,
а если конкретнее, работа аптеки.
Были выделены преимущества и недостатки конкурентных предложений.
На основе которых были определены основные функции, которые должна выполнять
система. Также были поставлены ожидаемые результаты.
И, в результате, была составлена диаграмма вариантов с актерами и связями.

\newpage

\begin{thebibliography}{0}
	\bibitem{1} \textit{Интернет-магазин аптеки 36.6}
		URL:~https://www.36.6.ru/
	\bibitem{2} \textit{Интернет-магазин аптеки Горздрав}
		URL:~https://gorzdrav.org/
\end{thebibliography}

