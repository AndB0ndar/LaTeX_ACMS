\graphicspath{{./sisxth/img/}} % path to graphics

\section*{\LARGE Цель практической работы}
\addcontentsline{toc}{section}{Цель практической работы}

\textbf{Цель работы} --- сформировать навык построения диаграммы компонентов
в Visual Paradigm.
\textbf{Задачи}:
\begin{itemize}
	\item ознакомиться с назначением диаграммы компонентов и ее элементами.
	\item изучить возможности Visual Paradigm для построения диаграммы
		компонентов.
	\item построить диаграмму компонентов в Visual Paradigm в соответствии с
		предложенными заданиями.
\end{itemize}


\clearpage

\section*{\LARGE Выполнение практической работы}
\addcontentsline{toc}{section}{Выполнение практической работы}
\section{Создание диаграммы компонентов}
Диаграммы компонентов используются для моделирования статического
вида системы с точки зрения реализации. Этот вид диаграмм в первую очередь
связан с управлением конфигурацией частей системы, составленной из
компонентов, которые можно соединять между собой различными способами.
Диаграмма компонентов (component diagram) описывает особенности
физического представления разрабатываемой системы, позволяя определить ее
архитектуру, установив зависимости между программными компонентами, в
роли которых могут выступать исходный, бинарный и исполняемый коды.
Данная диаграмма обеспечивает согласованный переход от логического к
физическому представлению системы в виде программных компонентов.\par
В данном случае (рис. \ref{fig:components}), в зависимости от роли
пользователя осуществляется вызов одной из подсистем: АРМ сотрудника
регистратуры или АРМ врача. Сотрудник регистратуры работает с подсистемой,
обрабатывающей информацию из двух баз данных: БД врачей и БД услуг.
Кроме того, на диаграмме показано, что терминал для выписывания талонов
на прием является интерфейсом.

\begin{image}
	\includegrph{Screenshot from 2023-04-30 16-36-18}
	\caption{Диаграмма компонентов}
	\label{fig:components}
\end{image}

Элементам обозначающим базы данных устанавливается стереотип «table».\par
Далее, создали компоненты "<Поисковая строка"> и "<Авторизация">.
На программном уровне данные компоненты представляют собой
подсистемы, которые пользователь, в зависимости от своего служебного статуса,
через главный интерфейс вызывает на исполнение. Поэтому задали
этим компонентам стереотип «subsystem».\par
Элементам "<Сайт">, "<Заказ">, "<Склад"> и "<Доставка">~--- 
имеют стереотип Executable (Исполнимый).

\clearpage

\section*{Ответы на вопросы}
\addcontentsline{toc}{section}{Ответы на вопросы}

\begin{description}
	\item [В чем состоит назначение диаграммы компонентов?]
		В моделировании статического вида системы с точки зрения реализации.

		Описать особенности физического представления разрабатываемой системы,
		позволяя определить ее архитектуру, установив зависимости между
		программными компонентами, в роли которых могут выступать исходный,
		бинарный и исполняемый коды.
	\item [Как создать отношения между компонентами диаграммы?]
		Для представления отношений используются пунктирные стрелки.
		Которые ведут либо от компонента к компоненту,
		либо от порта к интерфейсу.
	\item [Перечислите основные элементы диаграммы компонентов.]
		Компоненты, интерфейсы и отношения.
	\item [Что такое компонент?]
		\textbf{Компонент (component)} --- элемент модели,
		представляющий некоторую
		модульную часть системы с инкапсулированным содержимым, спецификация
		которого является взаимозаменяемой в его окружении.
	\item [Какие виды компонентов существуют?]
		Компоненты могут иметь следующие стандартные стереотипы:
		\begin{itemize}
			\item "<file"> --- любой файл кроме таблицы;
			\item "<executable"> --- программа (исполняемый файл);
			\item "<library"> --- статическая или динамическая библиотека;
			\item "<document"> --- остальные файлы (например, файл справки);
			\item "<table"> --- таблица базы данных.
		\end{itemize}
	\item [Что такое интерфейс?]
		Интерфейсы на компонентных схемах показывают, как компоненты
		соединены друг с другом и взаимодействуют друг с другом.
	\item [Как изображается интерфейс на диаграмме компонентов?]
		Предоставляемые и требуемые интерфейсы в языке UML 2.0 изображаются
		с использованием специальной нотации, которая получила название
		"<шара и шарнира"> или "<леденца на палочке">.
	\item [Какие виды и стереотипы определены для компонентов?]
		Компоненты могут иметь следующие стандартные стереотипы:
		\begin{itemize}
			\item "<file"> --- любой файл кроме таблицы;
			\item "<executable"> --- программа (исполняемый файл);
			\item "<library"> --- статическая или динамическая библиотека;
			\item "<document"> --- остальные файлы (например, файл справки);
			\item "<table"> --- таблица базы данных.
		\end{itemize}
	\item [Для чего на диаграмме компонентов используются зависимости?]
		Зависимости используются для связи компонентов с классификаторами,
		которые реализуют функциональность этого компонента.
\end{description}

\clearpage

\section*{\LARGE Вывод}
\addcontentsline{toc}{section}{Вывод}
Мы создали диаграмму компонентов.\par
При этом перед разработкой диаграмм компонентов решили, из каких
физических частей (файлов) будет состоять программная система.
А при спецификации общей структуры исходного кода системы
учитывали специфику языка программирования, с помощью
которого реализуются компоненты.\par
Также помнили, что на диаграмме могут быть представлены отношения
зависимости между компонентами и включенными в них классами.
Эта информация имеет важное значение для обеспечения согласованности
между логическим и физическим представлениями системы.

